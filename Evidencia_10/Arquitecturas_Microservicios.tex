
\documentclass[12pt]{article}
\usepackage[utf8]{inputenc} % Para caracteres especiales
\usepackage[spanish]{babel} % Configuración en español
\usepackage{amsmath} % Herramientas matemáticas
\usepackage{graphicx} % Insertar imágenes
\usepackage{geometry} % Configurar márgenes
\usepackage{hyperref} % Hipervínculos

% Configuración de márgenes
\geometry{a4paper, margin=1in}

% Información del título y autores
\title{Arquitecturas de Microservicios: Transformación y Perspectivas en el Desarrollo de Software}
\author{
Carlos Julio Cadena Sarasty\\
Manul Ricardo Díez Corredor\\
Jesús Ariel Gonzales Bonilla\\
\textit{Servicio Nacional de Aprendizaje SENA}\\
\texttt{manuelric1026@gmail.com}
}
\date{\today}

\begin{document}
\maketitle

\begin{abstract}
Este artículo explora la implementación y el impacto de las arquitecturas de microservicios como solución para los desafíos del desarrollo de software moderno. Basado en un análisis de 38 estudios, se examinan los principios teóricos, herramientas tecnológicas y metodologías ágiles como Scrum y Kanban que facilitan su adopción. Los resultados destacan mejoras significativas en escalabilidad, flexibilidad y tiempos de respuesta, así como desafíos relacionados con la complejidad operativa y la seguridad.
\end{abstract}

\section{Introducción}
En la actualidad, el desarrollo de software enfrenta el reto de adaptarse a entornos cada vez más dinámicos y exigentes, donde la escalabilidad, la flexibilidad y la eficiencia operativa son esenciales. Las arquitecturas de microservicios han surgido como una respuesta a estas necesidades, ofreciendo un enfoque modular y distribuido que permite a las organizaciones transformar sus procesos tecnológicos y optimizar sus recursos.

\section{Marco Teórico}
\subsection{Evolución de las Arquitecturas de Software}
Desde los años 70, las arquitecturas de software han cambiado considerablemente para adaptarse a los retos tecnológicos crecientes. \textit{(Incluye más detalles de tu sección)}.

\subsection{Microservicios: Conceptos y Principios Fundamentales}
Los microservicios dividen una aplicación en módulos pequeños e independientes, conocidos como servicios, cada uno dedicado a una función específica del sistema. Sus principios básicos son:
\begin{itemize}
    \item Independencia de los servicios.
    \item Comunicación a través de APIs.
    \item Despliegue independiente.
    \item Escalabilidad.
\end{itemize}

\subsection{Herramientas y Tecnologías Asociadas}
El desarrollo de microservicios ha sido impulsado por herramientas como:
\begin{itemize}
    \item Docker: Plataforma de contenedorización.
    \item Kubernetes: Orquestación de servicios.
\end{itemize}

\section{Metodología}
Para la elaboración de este estudio, se siguió un enfoque cualitativo y descriptivo basado en el análisis de 38 artículos relacionados con arquitecturas de microservicios. \textit{(Añade más detalles de tu metodología aquí)}.

\section{Resultados}
\begin{itemize}
    \item Incremento del 20-30\% en la velocidad de respuesta.
    \item Reducción de tiempos de desarrollo en un 40\%.
    \item Escalabilidad mejorada mediante módulos específicos.
\end{itemize}

\section{Conclusiones}
Las arquitecturas de microservicios representan un cambio significativo en el diseño y desarrollo de software, ofreciendo soluciones adaptables a las demandas de sistemas modernos. Sin embargo, su implementación inicial puede ser desafiante.

\section*{Referencias}
\begin{itemize}
    \item González, R. A., et al. (2023). "¿Son los Microservicios la Mejor Opción? Una Evaluación de su Eficacia y Eficiencia Frente a los Monolitos." Memorias de las 53 JAIIO.
    \item Saransig Chiza, A. F. (2018). "Análisis de rendimiento entre una arquitectura monolítica y una arquitectura de microservicios." Universidad Técnica del Norte.
    \item López, D., \& Amaya, E. (2017). "Arquitectura de Software Basada en Microservicios para el Desarrollo de Aplicaciones Web." Séptima Conferencia de Directores de Tecnología de Información.
\end{itemize}

\end{document}
